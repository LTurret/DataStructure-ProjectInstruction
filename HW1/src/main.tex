\documentclass[12pt]{report}

% Packages and configurations
\usepackage{amsmath}
\usepackage{amsfonts}
\usepackage{amssymb}
\usepackage[singlelinecheck=false]{caption}
\usepackage{enumerate}
\usepackage[shortlabels]{enumitem}
\usepackage{fancyhdr}
\usepackage[Lenny]{fncychap}
\usepackage{fontspec}
\usepackage[a4paper, top=2.5cm, bottom=2.5cm, left=3cm, right=2.5cm]{geometry}
\usepackage[pdf]{graphviz}
\usepackage{hyperref}
\usepackage{minted}
\usepackage{pgfplots}
\usepackage{tcolorbox}
\usepackage{titlesec}
\usepackage{tikz}
\usepackage{wrapfig}
\usepackage{xeCJK}

% Format Configuration
\pgfplotsset{compat=1.18}
\linespread{1.25}
\setlength{\parindent}{2em}
\setlength{\parskip}{1em}
\definecolor{LightGray}{HTML}{FAFAFA}
\captionsetup[figure]{font=footnotesize}
\makeatletter
\newcommand{\unchapter}[1]{
  \begingroup
  \let\@makechapterhead\@gobble
  \chapter{#1}
  \endgroup
}

% Header Configuration
\renewcommand{\chaptermark}[1]{\markboth{#1}{}}
\pagestyle{fancy}
\fancyhead[l]{\doctitle}
\fancyhead[r]{\assignment}
\fancyfoot[l]{\footnotesize \authorname}
\fancyfoot[c]{\footnotesize 第\hspace*{0.5em}\thepage\hspace*{0.5em}頁}
\fancyfoot[r]{\footnotesize \identify}
\renewcommand{\footrulewidth}{0.4pt}
\setlength{\headheight}{14.67484pt}
\addtolength{\topmargin}{-0.17485pt}

\fancypagestyle{plain}{
  \fancyhf{} % Clear all fields for plain pages
  \fancyhead[L]{\doctitle}
  \fancyhead[R]{\assignment}
  \fancyfoot[L]{\footnotesize \authorname}
  \fancyfoot[C]{\footnotesize 第\hspace*{0.5em}\thepage\hspace*{0.5em}頁}
  \fancyfoot[R]{\footnotesize \identify}
  \renewcommand{\footrulewidth}{0.4pt}
}

% Fonts Configuration
\setmainfont{Noto Serif CJK TC}
\setCJKmainfont{Noto Serif CJK TC}
\setCJKsansfont{Noto Serif CJK TC}
\setCJKmonofont{PingFang TC}
\setmonofont{FiraCode Nerd Font}
\CJKsetecglue{\hspace{-0.0005em}}

% Information
\title{\doctitle\ —\ \assignment}
\author{\authorname\\ \identify}
\date{\today}

\newcommand{\authorname}{作者}
\newcommand{\identify}{學號}
\newcommand{\doctitle}{資料結構}
\newcommand{\assignment}{HW}

\begin{document}

\maketitle
\tableofcontents

\chapter{解題說明}

\hspace{2em}以遞迴實作計算 $N$ 階層的函式,已知階層計算公式如下:

$$
N! = N+(N-1)!=N+(N-1)+\cdots +2+1
$$

由於我們得知遞迴函數屬性,因此實作程式如下:

\begin{figure}[ht]
    \begin{minted}[
        frame=single,
        framesep=2mm,
        baselinestretch=1.2,
        bgcolor=LightGray,
        fontsize=\footnotesize,
        linenos
    ]{cpp}
int sigma(int n) {
    if (n < 0)
        throw "n < 0";
    else if (n <= 1)
        return n;
    return n + sigma(n - 1);
}
    \end{minted}

    \captionsetup{justification=centering}
    \caption{sigma}
    \label{fig:sigma}
\end{figure}


\chapter{程式實作}

\begin{figure}[ht]
    \begin{minted}[
        frame=single,
        framesep=2mm,
        baselinestretch=1.2,
        bgcolor=LightGray,
        fontsize=\footnotesize,
        linenos,
        tabsize=4,
        breaklines
    ]{cpp}
#include <iostream>
using namespace std;

int sigma(int n) {
    if (n < 0)
        throw "n < 0";
    else if (n <= 1)
        return n;
    return n + sigma(n - 1);
}

int main() {
    int result = sigma(3);
    cout << result << '\n';
}
    \end{minted}
    \captionsetup{justification=centering}
    \caption{完整程式實作細節}
    \label{fig:完整程式實作細節}
\end{figure}

\chapter{效能分析}

$$f(n)=O(n)$$

\section*{時間複雜度}

$$T(P)=n\times C$$

每層迴圈所需 $C$ 時間、$n$ 次遞迴。

\section*{空間複雜度}

$$S(P)=1\times n$$

1 個變數、$n$ 次遞迴。

\chapter{測試與驗證}

\begin{figure}[ht]
    \begin{minted}[
        frame=single,
        framesep=2mm,
        baselinestretch=1.2,
        bgcolor=LightGray,
        fontsize=\footnotesize,
        linenos
    ]{cpp}
#include <iostream>
using namespace std;

int sigma() {...}

int main() {
    int result = sigma(3);
    cout << result << '\n';
}
    \end{minted}

    \captionsetup{justification=centering}
    \caption{主函式細節}
    \label{fig:主函式細節}
\end{figure}

\begin{figure}[ht]
    \begin{minted}[
        frame=single,
        framesep=2mm,
        baselinestretch=1.2,
        bgcolor=LightGray,
        fontsize=\footnotesize,
        linenos
    ]{shell}
$ g++ main.cpp -o main.exe && ./main.exe
6
    \end{minted}

    \captionsetup{justification=centering}
    \caption{shell 編譯指令與輸出結果}
    \label{fig:shell 編譯指令與輸出結果}
\end{figure}

此函式遞迴終止條件為當 $n$ 為 $0$ 或 $1$,若欲求得 $3!$,則呼叫 $sigma(3)$,進入函式後,首先第一層 $n = 3 > 1$ 所以回傳 $n + sigma(n − 1)$,即 $3 + sigma(2)$,接著第二層計算 $sigma(2)$,$n = 2 > 1$,所以回傳 $2 + sigma(1)$,接下來到第三層時,$n = 1 \le 1$,符合終止條件 $(n ≤ 1)$,因此回傳 $n$,即 $1$。

$$sigma(3)=3+sigma(2)=3+2+sigma(1)=3+2+1=6$$

\chapter{申論及心得}

您的報告內容於此$\dots$

\end{document}