\documentclass[12pt]{report}

\usepackage{amsmath}
\usepackage{amsfonts}
\usepackage[singlelinecheck=false]{caption}
\usepackage{fancyhdr}
\usepackage[Glenn]{fncychap}
\usepackage[a4paper, top=2.5cm, bottom=2.5cm, left=3cm, right=2.5cm]{geometry}
\usepackage[pdf]{graphviz}
\usepackage{hyperref}
\usepackage{minted}
\usepackage{pgfplots}
\usepackage{tcolorbox}
\usepackage{tikz}
\usepackage{wrapfig}
\usepackage{xeCJK}

% Informations.
\title{資料結構報告範例}
\author{姓名}
\date{\today}

% Fomat Configuration.
\linespread{1.25}
\setlength{\parindent}{2em}
\setlength{\parskip}{1em}
\definecolor{LightGray}{HTML}{FAFAFA}

% Header Configuration.
\renewcommand{\chaptermark}[1]{\markboth{#1}{}}
\fancypagestyle{plain}{
    \fancyhead{}
    \fancyhead[l]{資料結構}
    \fancyhead[c]{課堂作業}
    \fancyhead[r]{\today}
    \fancyfoot[l]{姓名}
    \fancyfoot[c]{第 \thepage 頁}
}
\pagestyle{fancy}
\fancyhead[l]{資料結構}
\fancyhead[c]{課堂作業}
\fancyhead[r]{
    \nouppercase{
        \scshape\footnotesize\chaptername\ 
        \large\thechapter\ 
        \normalsize\leftmark
    }
}
\fancyfoot[l]{姓名}
\fancyfoot[c]{第 \thepage 頁}
\setlength{\headheight}{14.67484pt}
\addtolength{\topmargin}{-0.17485pt}

% Chinese Font Configuration.
\setmainfont{Noto Serif CJK TC}
\setCJKmainfont{Noto Serif CJK TC}
\setCJKsansfont{Noto Serif CJK TC}


\begin{document}

\maketitle
\tableofcontents
\clearpage

% Begin your writing here!
\chapter{解題說明}

\hspace*{2em}以遞迴實作計算 $N$ 階層的函式,已知階層計算公式如下:

$$N! = N + (N-1)! = N + (N-1) + \dots + 2 + 1$$

實作參見檔案 sum.cpp,其遞迴函式:

\begin{figure}[ht]
    \begin{minted}[
        frame=single,
        framesep=2mm,
        baselinestretch=1.2,
        bgcolor=LightGray,
        fontsize=\footnotesize,
        linenos
    ]{cpp}
int sigma(int n){
    if(n<0) throw "n < 0";
    else if(n<=1) return n;
    return n+sigma(n-1);
}
    \end{minted}

    \captionsetup{justification=centering}
    \caption{sum.cpp}
    \label{fig:sum.cpp}
\end{figure}

\chapter{演算法設計與實作}

\begin{figure}[ht]
    \begin{minted}[
        frame=single,
        framesep=2mm,
        baselinestretch=1.2,
        bgcolor=LightGray,
        fontsize=\footnotesize,
        linenos
    ]{cpp}
int main(){
    int n=0;
    while(cin>>n){             // 一直輸入
        if(n<=0) break;        // 直到輸入小於等於零
        cout<<sigma(n)<<"\n";  // 依據每次輸入,輸出 n!
    }
    return 0;
}
    \end{minted}

    \captionsetup{justification=centering}
    \caption{main.cpp}
    \label{fig:main.cpp}
\end{figure}

\chapter{效能分析}

$$f(n) = O(n)$$

\section*{時間複雜度}

$$T(P) = n \times C$$
每層迴圈所需 $C$ 時間、$n$ 次遞迴。

\section*{空間複雜度}

$$S(P) = 1 \times n $$
$1$ 個變數、$n$ 次遞迴。

\chapter{測試與過程}

\begin{figure}[ht]
    \begin{minted}[
        frame=single,
        framesep=2mm,
        baselinestretch=1.2,
        bgcolor=LightGray,
        fontsize=\footnotesize,
        linenos
    ]{powershell}
$ g++ main.cpp -o main.exe && ./main.exe
3 7 11
6
28
66
    \end{minted}

    \captionsetup{justification=centering}
    \caption{shell command}
    \label{fig:shell command}
\end{figure}

\section*{驗證}

\hspace*{2em}此函式遞迴終止條件為當 $n$ 為 $0$ 或 $1$,若欲求得$3!$,則呼叫 $sigma(3)$,進入函式後,首先第一層 $n = 3 > 1$ 所以回傳 $n + sigma(n - 1)$,即 $3 + sigma(2)$,接著第二層計算 $sigma(2)$,$n = 2 > 1$,所以回傳 $2 + sigma(1)$,接下來到第三層時,$n = 1 \le 1$,符合終止條件 $(n \le 1)$,因此回傳 $n$,即 $1$。

$$sigma(3) = 3 + sigma(2) = 3 + 2 + sigma(1) = 3 + 2 + 1 = 6$$

\end{document}